% latex 2.09 header
%\documentstyle[twoside,titlepage,fullpage,fancyheadings]{article}

% latex2e header
\documentclass[11pt]{article}
\usepackage{latexsym}
\usepackage{fullpage}
%\usepackage{fancyheadings}
%\usepackage{pstricks}
%\usepackage{psfig}
%\pagestyle{fancyplain}


%\input{lab-fmt.tex}
\input epsf.sty
\setlength{\textheight}{10.0in}
\begin{document}
\thispagestyle{empty}
Professor Fearing ~~~~~~~~~~ EECS192/Assignment \#2 Car Steering~~~~~~~ Spring 2015\\
{\bf Due by 5 pm Fri. April 3 in 725 S.D. Hall.} \\

(One assignment per team.) 

For this assignment you will be using  {\tt V-rep} for dynamic simulation
of a car. (V-rep Pro EDU V3.2 
is available in 204 Cory, and can be downloaded for Mac/Ubuntu/Windows at \\
{\tt http://www.coppeliarobotics.com/downloads.html}. 
The car model {\tt car-scale.ttt} is on Piazza. V-rep is the
server, and you will write a Python remote API to control the simulator.
The simulator estimates ${\tt yDist} = y_a$,
 the lateral error from the track
at a distance approximately 2 car lengths in front, and takes
2 inputs: 1)  commanded steering angle ${\tt steerAngle} = \delta$
and 2) longitudinal velocity ${\tt xSpeed}= V$. 
See the {\tt readMe.txt} in the Python directory. The V-rep simulation
server needs to start before {\tt carTest.py}.
{\tt carTest.py} will run the car but it is not tuned.
You will modify and extend the code to get a good car controller.
The simulator runs ok at 10 ms time step (dynamics are updated at 10X).

For all plots, time axes should be in seconds, and 
lateral errors in m or cm.
Do not use a white-on-black screen dump (as it looks unprofessional).
Matplotlib (see iPython notebook from hw1) is suggested to 
make nice plots.

For each part below, plot on 1 page 
a) actual x vs y position of car b) lateral error 
$y_a$ as function of time, c) steering angle $\delta$ as function of time.
For part 4, below, also plot d) car longitudinal velocity as a function of 
time.
For each part, list constants used. 
($k_p$ should have units of radians/meter, etc.)\\

(25 pts) 1. Steering Simulation- proportional control\\
(22) a. Using pure position control, $\delta = k_p y_a$, choose a fixed speed
$V$ and $k_p$ that allows the car to successfully complete the track
without hitting any cone(s). Report $k_p$ and $V$.\\
(3) b. Specify worst-case overshoot (cm), and note on plots where this occurs.\\

(25 pts) 2. Steering Simulation- proportional + derivative control\\
(20) a. Using PD control $\delta = k_p y_a + k_d \dot y_a$, choose
a fixed speed that allows the car successfully complete the track
without hitting any cone(s). Report $k_p$, $k_d$, and $V$.
(Estimate $\dot y_a \approx \frac{y_a[n] - y_a[n-1]}{20 ms}$.)\\
(3) b. Specify worst-case overshoot (cm), and note on plots where this occurs.\\
(2) c. Briefly comment on any differences in performance observed
between P and PD type control, such as overshoot or maximum speed.\\

(30 pts) 3. Sensor Resolution Effect\\
(10) a. For your line camera mounting location and orientation,
estimate track position resolution $\Delta$ (in cm) which is limited by the 128
elements. (Optionally, use your algorithm from HW1).  
Standard deviation due to position quantization can be estimated as
$\sigma = \frac{\Delta}{\sqrt{12}}$.\\
(20) b. Repeat problem 2 above, but adding the noise into 
estimated $y_a$.\\
i.e. {\tt  yDist = yDist + np.random.normal(0.0, sigma)}.\\

(20 pts) 4. Steering simulation with PD, quantization, and speed control\\
(15) a. Modify {\tt carTest.py} to use
car speed control rather than a fixed speed.
For example, decelerate when the lateral error is large, and speed
up when tracking well. (Or choose a velocity setpoint depending
on error.) Plot best car performance. Specify the velocity
controller used.\\
(5) b. What is worst case overshoot? What is laptime? Has the 
best time improved compared to fixed car speed?\\
\end{document}

