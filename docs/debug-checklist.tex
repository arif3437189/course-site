\documentclass[11pt]{article}
\usepackage{latexsym}
\usepackage{fullpage}
\usepackage{fancyheadings}
\usepackage{pstricks}
%\usepackage{psfig}
\pagestyle{plain}
\setlength{\textheight}{9in}

%\input{lab-fmt.tex}
\input epsf.sty
\begin{document}
\thispagestyle{plain}
Professor Fearing~~~~~~~~~~~~~EECS192 Debugging Checklist ~~v1.01~~~~~~~~~~Spring 2015


\section*{\small Debugging Commandments}
{\em Anything that can go wrong will go wrong. }
Which means that you need to follow this checklist every time
you want to debug. (adapted from 
{\tt www-inst.eecs.berkeley.edu/\~\ cs150/fa98/labs/debug/debug.html })

\begin{enumerate}
\item
Collect documentation.\\
Don't use processor bandwidth remembering design details. Save processor 
bandwidth for observations and questions.
Have schematics, board layout, wiring diagrams, data sheets, documented
code and any needed manuals open and ready use.

\item
Visually Inspect the Board/Setup\\
A quick check to verify that chips are still plugged in,
no wires have broken, or pins bent, is worth a minute,
before you start, but is unlikely to find most bugs.
\item
Check Software Functionality\\
Is the computer system and network functioning? Are your
files missing pieces or corrupted? (You do have memory stick backups, right?)
Verify that the software tools are working as expected.
Test a known working simulation/example; do the tools still work?
\item
Check Hardware Functionality\\
Are all cables, oscilloscope and logic probes working?
(You can check the oscilloscope probes by using the Calib output
on the front of the oscilloscope. You should see a nice square wave).
Is test equipment set up for desired sweep rate, volts-per-division,
etc. (Please notify your TA about bad cables so other students
won't have to find them the hard way).
Do you really have power supply voltages correct (Verify with scope)?
Are grounds tied together for battery, CPU, power supply, and test equipment,
or are you relying on the AC power line to provide a common ground?
Verify that basic functionality by testing a known good 
circuit. 

\item
Check System Essentials\\
Is the clock running at the expected frequency?
Do voltage levels seem reasonable (using oscilloscope to measure)?

\item
Divide and Conquer

Isolate modules and debug each module independently.

\item 
Scientific Debugging

The four keys to scientific debugging are:
\begin{enumerate}
\item 
Model Expected Behavior. What do you expect to see?
Simulators tell you what you have, not what you need to see.
\item
Measure the real system's behavior using oscilloscope. 
Where are the differences from your model significant?
\item
Hypothesize possible errors.
\item
Controllability and Observability. Force module inputs
to verify hypotheses, and observe outputs. Are outputs consistent
with your model, or is something not functioning as expected?
(Various test cases- DC input, AC input, single stepping).

\end{enumerate}

\item
Understand Component Devices\\
Are you sure you understood the data sheet correctly on
that part?

\item
Know your Software\\
Is the software working as you expect it to? 
Is there a mismatch between your expectations and the result?
Are there hidden settings/state which changed from a previous session?
Are you using the right commit?

\item
Know Your Test Equipment\\
The digital scope will give nonsense if the sweep rate is
too high or too low. 
When in doubt about a signal,
verify that you have proper voltage levels and timing
using the oscilloscope.

\item
Take a Break/Get a New Perspective\\
Debugging is mentally taxing, and we tend to focus in
narrowly on particular suspect areas. Taking a break opens
the mind to a wider range of possibilities and can get
you out of a rut.

\end{enumerate}

\end{document}



