\documentclass[10pt]{article}
\oddsidemargin  -.25in
\evensidemargin -.25in
\textwidth      6.0in
\topmargin      -0.25in
\textheight     8.5in
\parskip        0.0675in
\parindent      0.0in
\pagestyle{empty}
\begin{document}

{\Large \bf EECS192 Mechatronic Design Laboratory- Spring 2015}
(1/17/2015)

Instructor: Prof. R. Fearing, Office 725 Sutardja Dai Hall, x2-9193.
\\
Office Hours: Mon 130-230 pm, Thu 200-300pm \\
Please email for an appointment at another time (ronf@eecs.berkeley.edu).

TA: Richard Lin,  richard.lin@berkeley.edu Office hours (tba) in 204 Cory.\\
Class meeting: Tue 0930-1100 am 293 Cory Hall.\\
Lab lecture/demo Wed 9-10 or Th 11-12, 204 Cory Hall.
Checkoffs tentatively Fridays, time tbd. At least one team member must
be present to demonstrate functionality.

Grading: 15\% checkoffs, 20\% final exam, 18\% oral and written reports, 
8\% written assignments, 10\% first round contest, 
20\% second round contest,  4\% community spirit, 5\% in class 10 minute 
quizzes.

Recommended Texts: (on reserve in Engineering Library)
{\em Mechatronics: mechanical system interfacing} by D.M. Auslander;
{\em Analytical robotics and mechatronics} by W. Stadler;
{\em Robotic engineering: an integrated approach} by R.D. Klafter;
{\em The Art of Electronics} by Horowitz and Hill;
\\
Suggested reference: {\em Introduction to Mechatronic Design} 
by J.E. Carryer, R.M. Ohlnie, and T.W. Kenny.
Please check the class web page: www-inst.eecs.berkeley.edu/\~{}ee192
for class handouts and pointers to data sheets, etc. Also, announcements
and discussion will be on {\tt piazza}.
\\




\footnotesize
\begin{tabular}{l| c|l|l}
\hline
wk & lecture & Lecture and Demo Topics & Project Checkpoint\\
\hline
1& 1/20 & proj. description, ARM Cortex M0 overview, & team formation\\
&  & peripheral interface \\
&  & Demo: soldering I, ARM Cortex M0, car \\
\hline
2& 1/27 & motors, motor control, CortexM0 IO & Hello World, LED blink\\
 & & electronic construction practices & \\
 &  & Demo: soldering II, Keil, Eagle, test equipment \\
\hline
3& 2/3 & PWM, H Bridge, power MOSFET & written project proposal Fri. Feb. 6\\
 & & Demo: RC servo, motor circuit and waveforms & car clean and checked\\
\hline
4 & 2/10 & RC servo, CortexM0 PWM, Power Supply I & CPU turns motor on/off 
(on bench - stalled)\\
 & & Demo: switching power supply waveforms 
               & CPU turns front wheel left/right\\
\hline
5& 2/17  & Power Supply II  & drive motor from battery\\
&  & Demo: power filtering, PCB & power PCB (date tbd) \\ 
\hline
6& 2/24  & optical encoder, velocity sensing & to be determined \\
&  & Demo: velocity control, speed sensor &  \\
\hline
7&3/3 & line sense intro & drop and run test, open loop Figure 8 (PCB on car) w/e-stop\\
& & Demo: optical line sensing & {\bf lab clean}\\
\hline
8&3/10 & steering control I, line detection & bench top line following, drop and run\\
& & Demo: steering control\\
\hline
9&3/17 & steering II and velocity control  & closed loop Figure 8 line following I, drop and run\\
& & Demo: Matlab control and step responses & (outside track setup) \\
& & & assignment \#1 due Tues 3/17\\
 \hline
 & 3/23 & Spring Break & Spring Break\\
\hline
10 & 3/31 & CT and DT control & velocity control, Figure 8 ($>$ 1 m/sec), 
sensor mech. response,\\
& & demo: step response, car running &  {\bf lab clean}  \\
&  &                     & Assignment \#2 due Fri. 4/3\\
\hline
11 & 4/7 & feedforward control and filtering & practice course and step response\\
 &  & Demo: speed adaptation & Progress report due Tues. Apr. 7\\
\hline
12 &4/14 & HW and SW robustness & Round 1: Mon 4/13\\
 & &                        & CAL Day/ UC Davis Freescale Cup Sat April 18 \\
\hline
13 & 4/21 & Mechatronic system examples I &\\
& &   \\ 
\hline
14 & 4/28 & Mechatronic system examples II &  Round 2: Mon 4/27, {\bf lab clean}\\
\hline
& 4/30, 5/1  & Student Oral Reports (Th/ Fri)\\
\hline
& 5/5 & optional Final Review\\
\hline
 & 5/13 & final exam Wed. May 13, 1130 -230 pm &\\
\hline
& 5/23 & (Sat.) optional NATCAR contest (UC Davis) & \\
\hline
\end{tabular}
\par
\end{document}
